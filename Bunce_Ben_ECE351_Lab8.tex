%%%%%%%%%%%%%%%%%%%%%%%%%%%%%%%%%%%%%%%%%%%%%%%%%%%%%%%%%%%%%%%%
% %
% Ben Bunce %
% ECE351 - 52 %
% Lab 8 %
% 9.30.2021 %
% %
%%%%%%%%%%%%%%%%%%%%%%%%%%%%%%%%%%%%%%%%%%%%%%%%%%%%%%%%%%%%%%%%

%%% DOCUMENT PREAMBLE %%%
\documentclass[12pt]{report}
\usepackage[english]{babel}
%\usepackage{natbib}
\usepackage{url}
\usepackage[utf8x]{inputenc}
\usepackage{amsmath}
\usepackage{graphicx}
\graphicspath{{images/}}
\usepackage{parskip}
\usepackage{fancyhdr}
\usepackage{vmargin}
\usepackage{listings}
\usepackage{hyperref}
\usepackage{xcolor}

\setmarginsrb{3 cm}{2.5 cm}{3 cm}{2.5 cm}{1 cm}{1.5 cm}{1 cm}{1.5 cm}

\title{Fourier Series Approximation of a Square Wave}								
% Title
\author{Ben Bunce}						
% Author
\date{10/28/2021}
% Date

\makeatletter
\let\thetitle\@title
\let\theauthor\@author
\let\thedate\@date
\makeatother

\pagestyle{fancy}
\fancyhf{}
\rhead{\theauthor}
\lhead{\thetitle}
\cfoot{\thepage}
%%%%%%%%%%%%%%%%%%%%%%%%%%%%%%%%%%%%%%%%%%%%
\begin{document}

%%%%%%%%%%%%%%%%%%%%%%%%%%%%%%%%%%%%%%%%%%%%%%%%%%%%%%%%%%%%%%%%%%%%%%%%%%%%%%%%%%%%%%%%%

\begin{titlepage}
	\centering
    \vspace*{0.5 cm}
   % \includegraphics[scale = 0.075]{bsulogo.png}\\[1.0 cm]	% University Logo
\begin{center}    \textsc{\Large   ECE 351 - Section 52}\\[2.0 cm]	\end{center}% University Name
	\textsc{\Large Lab 8  }\\[0.5 cm]				% Course Code
	\rule{\linewidth}{0.2 mm} \\[0.4 cm]
	{ \huge \bfseries \thetitle}\\
	\rule{\linewidth}{0.2 mm} \\[1.5 cm]
	
	\begin{minipage}{0.4\textwidth}
		\begin{flushleft} \large
		%	\emph{Submitted To:}\\
		%	Name\\
          % Affiliation\\
           %contact info\\
			\end{flushleft}
			\end{minipage}~
			\begin{minipage}{0.4\textwidth}
            
			\begin{flushright} \large
			\emph{Submitted By :} \\
			Ben Bunce  
		\end{flushright}
           
	\end{minipage}\\[2 cm]
	
%	\includegraphics[scale = 0.5]{PICMathLogo.png}
    
    
    
    
	
\end{titlepage}

%%%%%%%%%%%%%%%%%%%%%%%%%%%%%%%%%%%%%%%%%%%%%%%%%%%%%%%%%%%%%%%%%%%%%%%%%%%%%%%%%%%%%%%%%

\tableofcontents
\pagebreak

%%%%%%%%%%%%%%%%%%%%%%%%%%%%%%%%%%%%%%%%%%%%%%%%%%%%%%%%%%%%%%%%%%%%%%%%%%%%%%%%%%%%%%%%%
\renewcommand{\thesection}{\arabic{section}}

\section{Introduction}
Using python, the Fourier series of a square wave is found and then plotted for a number of N values.

\section{Equations}
\\$w = \frac{2\pi}{T}$
\\
\\$a_k = 0$
\\
\\$b_k = 2[\frac{1-cos(\pi k)}{\pi k}]$
\\
\\$a_0 = 0$
\\
\\$x(t) = \sum_{n=1}^{\infty} [\frac{2(1-cos(\pi k))sin(kwt)}{\pi k}]$



\section{Methodology}
First, the equations found in the prelab are inputted into python. In python, multiple n values are inputted into the equations and the outputs are printed. Next, the Fourier series function is plotted for $N = {1,3,15,50,150,1500}$, with $T=8s$. 



\section{Results}
Task 1 Outputs:
\\
\\ \includegraphics[width=3in]{anresult.PNG}
\\
\\ Plot Set 1:
\\
\\ \includegraphics[width=3in]{plot2.PNG}
\\
\\ Plot Set 2:
\\
\\ \includegraphics[width=3in]{plot1.PNG}


\section{Questions}

\textbf{1. Is x(t) an even or odd function? Explain}
\\x(t) is an even function because x(-t) = -x(t), which is defined as an even function.
\\
\\ \textbf{2. Based on results from Task 1, what do you expect a1, a2, ..., an to be?}
\\All values of $a_n$ will be equal to 0. This is because for odd function, $a_n = 0$ for all values.
\\
\\ \textbf{3. How does the approximation of the square wave change as the value of N increases? In what way does the Fourier series struggle to approximate the square wave}
\\As the value of N increases, the approximation of the square wave gets more accurate. The Fourier series struggles to approximate the square wave at low N values, where the wave turns out inaccurate.
\\
\\ \textbf{4. What is occurring mathematically in the Fourier series summation as the value of N increases?}
\\As N increases, the number of times the Fourier series is added increases, which leads to greater accuracy of the Fourier series.

\section{Conclusion}
In this lab, python was used to generate plots of the Fourier series approximation of a square wave. In the plots, it is shown that as the N value increases, the accuracy of the approximation increases, to the point where, at $N = 1500$, the Fourier series looked identical to the square wave function.


\end{document}



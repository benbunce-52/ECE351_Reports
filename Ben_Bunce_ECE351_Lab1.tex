%%%%%%%%%%%%%%%%%%%%%%%%%%%%%%%%%%%%%%%%%%%
%%% DOCUMENT PREAMBLE %%%
\documentclass[12pt]{report}
\usepackage[english]{babel}
%\usepackage{natbib}
\usepackage{url}
\usepackage[utf8x]{inputenc}
\usepackage{amsmath}
\usepackage{graphicx}
\graphicspath{{images/}}
\usepackage{parskip}
\usepackage{fancyhdr}
\usepackage{vmargin}
\usepackage{listings}
\usepackage{hyperref}
\usepackage{xcolor}

\definecolor{codegreen}{rgb}{0,0.6,0}
\definecolor{codegray}{rgb}{0.5,0.5,0.5}
\definecolor{codeblue}{rgb}{0,0,0.95}
\definecolor{backcolour}{rgb}{0.95,0.95,0.92}

\lstdefinestyle{mystyle}{
    backgroundcolor=\color{backcolour},   
    commentstyle=\color{codegreen},
    keywordstyle=\color{codeblue},
    numberstyle=\tiny\color{codegray},
    stringstyle=\color{codegreen},
    basicstyle=\ttfamily\footnotesize,
    breakatwhitespace=false,         
    breaklines=true,                 
    captionpos=b,                    
    keepspaces=true,                 
    numbers=left,                    
    numbersep=5pt,                  
    showspaces=false,                
    showstringspaces=false,
    showtabs=false,                  
    tabsize=2
}
 
\lstset{style=mystyle}

\setmarginsrb{3 cm}{2.5 cm}{3 cm}{2.5 cm}{1 cm}{1.5 cm}{1 cm}{1.5 cm}

\title{1}								
% Title
\author{ Your name}						
% Author
\date{today's date}
% Date

\makeatletter
\let\thetitle\@title
\let\theauthor\@author
\let\thedate\@date
\makeatother

\pagestyle{fancy}
\fancyhf{}
\rhead{\theauthor}
\lhead{\thetitle}
\cfoot{\thepage}
%%%%%%%%%%%%%%%%%%%%%%%%%%%%%%%%%%%%%%%%%%%%
\begin{document}

%%%%%%%%%%%%%%%%%%%%%%%%%%%%%%%%%%%%%%%%%%%%%%%%%%%%%%%%%%%%%%%%%%%%%%%%%%%%%%%%%%%%%%%%%

\begin{titlepage}
	\centering
    \vspace*{0.5 cm}
   % \includegraphics[scale = 0.075]{bsulogo.png}\\[1.0 cm]	% University Logo
\begin{center}    \textsc{\Large   ECE 351 - Section \# }\\[2.0 cm]	\end{center}% University Name
	\textsc{\Large Lab Title  }\\[0.5 cm]				% Course Code
	\rule{\linewidth}{0.2 mm} \\[0.4 cm]
	{ \huge \bfseries \thetitle}\\
	\rule{\linewidth}{0.2 mm} \\[1.5 cm]
	
	\begin{minipage}{0.4\textwidth}
		\begin{flushleft} \large
		%	\emph{Submitted To:}\\
		%	Name\\
          % Affiliation\\
           %contact info\\
			\end{flushleft}
			\end{minipage}~
			\begin{minipage}{0.4\textwidth}
            
			\begin{flushright} \large
			\emph{Submitted By :} \\
			Student Name  
		\end{flushright}
           
	\end{minipage}\\[2 cm]
	
%	\includegraphics[scale = 0.5]{PICMathLogo.png}
    
    
    
    
	
\end{titlepage}

%%%%%%%%%%%%%%%%%%%%%%%%%%%%%%%%%%%%%%%%%%%%%%%%%%%%%%%%%%%%%%%%%%%%%%%%%%%%%%%%%%%%%%%%%

\tableofcontents
\pagebreak

%%%%%%%%%%%%%%%%%%%%%%%%%%%%%%%%%%%%%%%%%%%%%%%%%%%%%%%%%%%%%%%%%%%%%%%%%%%%%%%%%%%%%%%%%
\renewcommand{\thesection}{\arabic{section}}
\section{Introduction}
 

The paper will look professional and will be typed in \LaTeX. Make sure the cover page includes the project title, your name, and the date. The introduction should restate the goal of the lab in your own words and provide any necessary background information including the tools or software(s) used to complete the lab. \textit{Note: the title page should not be numbered or included in the table of contents and headers need to start on the first page with actual content.}

\section{Equations}

This section needs to have every equation or mathematical definition used to complete the lab. This needs to be written using an equation writing environment and not just added into text. You can use equation numbers if you would like, but they are not necessary. Equation numbers allow for easy referencing of the equation later in the report, however. An example of an equation that is numbered and one without equation numbers are provided below.
\begin{equation}
    V = I\cdot R
\end{equation}
\begin{equation*}
    P = V\cdot I
\end{equation*}

\section{Methodology}

This section will describe how you went about solving the lab. Make sure you go into detail about any method you used. Include coding samples here if necessary. This is also where you would include necessary derivations. An example of inserting code into the report is given. Do not go overboard on inserting code into your report, only use whats absolutely necessary to illustrate your point.
\begin{lstlisting}[language=Python]
#%% Example Code for ECE 351 Sample Report

I = 15e-3 #15 mA
R = 1e3 #1 kOhm

V = I*R
P = V*I

print('Twinkle Twinkle Little Star V is equal to IR: ', V, 'V')
print('\nLike a Diamond in the Sky P is equal to VI: ', P, 'W')
\end{lstlisting}

\section{Results}

This section will go over the results of the lab. Use this area to describe if the lab worked as expected or if the results are unexpected or different from your hand calculations or intuition. Part of being a good engineer is gaining intuition about these problems and being able to understand quickly if something is wrong. Use code, plots, tables, and figures as necessary. Make sure to cite all other works used and note them in the bibliography. A sample entry is in this document.

\section{Error Analysis}

This section will discuss error analysis of the experiment. Since this lab deals with ideal simulation there shouldn't be any sources of error, so instead this section can be used to describe any difficulties you had during lab and how you solved them. Alternatively, if you couldn't get the experiment to work, which is ok, you need to use this section to explain why you couldn't get it to work to earn full points. 

\section{Questions}

Use this section to answer ALL questions in the lab handout and to address any other deliverables that aren't covered in the other sections. 

\section{Conclusion}

Discuss briefly what you learned in this lab and whether or not you feel the lab was successful. Include any recommendations for future labs as this is a learning experience for all of us. Discuss any insights you gained from this lab and how that will affect future work. \textit{Note: The bibliograhpy needs to be on its own page.}

\newpage


\begin{thebibliography}{111}

  \bibitem{ACMT}
A. Aldroubi, C. Cabrelli, U. Molter, and Sui Tang,
Dynamical sampling, 
{\it  Applied and Computational Harmonic Analysis}, doi:10.1016/j.acha.2015.08.014, 2016


\end{thebibliography}
\end{document}

%This template was created by Roza Aceska.